%!TEX root = paper.tex
\begin{abstract}

In this paper we present Seattle, a practical and publicly accessible
platform with considerable deployment history.
Seattle's sandbox implementation solves the widely-recognized issue
of node heterogeneity in \gls{fc}, and 


het = [nodes, networks, ops]

and deployment models.
Seattle's Python-based sandbox runs on multiple
operating systems and platforms.
Our system architecture supports heterogeneous
deployment models, from isolated/standalone and peer-to-peer to
full-fledged provisioning by a dedicated operator.

\albert{Mention the term ``federation''.}

% This characteristic, while taken
%for granted in Internet services, but often overlooked in the context
%of \gls{fc}.
Seattle's components and interfaces are
designed for compatibility and reuse, and align with the trust boundaries
that exist between different stakeholders.

Besides our own use of Seattle as a distributed network testbed for
teaching and research, outside groups have used existing Seattle
components, and constructed new components with compatible interfaces,
to adapt the platform to their needs. This has resulted in edge node
selection tools based on social graphs, an Android-compatible computing
sandbox that can access smartphone sensors, \albert{etc.}

Seattle is FOSS, .... ?

\end{abstract}
