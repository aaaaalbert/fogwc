%!TEX root = paper.tex
\section{Implementation}

In this section we will look at the actual implementations of
selected components, with a focus on those aspects that help
Seattle solve the node and operational heterogeneity
challenges encountered in \gls{fc}.

\subsection{Sandbox Implementation}

% Inspired by the repy_v2/README that I wrote a few weeks back.
Seattle's Python-based default sandbox~\cite{RepySandbox}
offers a cross-platform portable, resource-isolated,
safe execution environment for untrusted experimenter code.

The choice of a high-level programming language environment trades
off some reduction in performance against large gains in portability
across \glspl{OS} and devices. Seattle's platform abstractions not only
provide a unified \gls{API} to sandboxes on desktops and laptops,
but also smartphones and even WiFi routers. Seattle sandboxes
run on Windows, Mac OS X and later, Linux, and the \glspl{BSD}, as
well as Android and OpenWrt.

Seattle's resource isolation scheme~\cite{li2015fence} ensures
that each sandbox is confined to strict usage quotas for all
resources of the hosting system, including its \gls{CPU} time and
memory, used disk space, and even \gls{IP} addresses and port numbers
on network interfaces.

In terms of code safety, Seattle keeps buggy or deliberately destructive
experimenter code from harming the host machine. This is guaranteed
by first checking the static code safety and forbidding potentially
problematic statements (like importing arbitrary modules). At runtime,
all of the \gls{API} functions strictly check their call parameters,
so that, for example, file names can be sanitized and restricted to the
sandbox directory.

Other sandbox implementations have augmented Seattle's basic functionality.
Sensibility Testbed adds sensor functions to Seattle's Python-based
sandbox. Another internal research prototype of ours interfaces the
Seattle Resource Manager with Docker, so that its sandboxes are
Linux Containers.



%\subsection{Network Heterogeneity Masking}
%
%Nodes behind firewalls or \gls{NAT} gateways are common in
%today's networks. In order to make sandboxes on such nodes
%accessible for experiment use, Seattle implements \gls{NAT}
%traversal and locator--identifier split mechanisms.
%For the first, we run custom \gls{NAT} relays on public Seattle nodes



\subsection{Component Interface Implementation}

A core solution to flexibly addressing operational heterogeneity (see
Section~\ref{sec-op-flex}) lies in clearly-defined interfaces
that use simple, text-based, mostly stateless protocols.
Such an approach has a few benefits. Code handling communications between
components can be much simpler, as parsing is easy to verify,
even if done manually. As such, it is much easier to reimplement these protocols
to interface with Seattle from other systems. Lastly, keeping
and tracking state in a distributed fashion is a notoriously
difficult problem that we attempt to avoid wherever possible.

For communication that requires a token of privilege to trigger
an action on the remote side, control messages contain signatures
constructed with the private key of the experimenter. A
sandbox controlled by that experimenter knows the
corresponding public key (as it previously downloaded a Seattle installer
customized to contain this key). Only after checking
the message signature, are privileged operations, such as starting
or stopping the sandbox or transferring data to/from its
local storage, permitted.

Scripted (i.e. non-interactive) communications with the
Clearinghouse, Installer Builder, and Software Updater all
use \gls{HTTP} or \gls{HTTPS} in the form of an
\acrshort{XML}-\acrshort{RPC} interface, which further simplifies
integration. Also, these interfaces
follow what is known as Postel's Robustness
Principle~\cite[\S2.10]{rfc793}, and are quite liberal in the
data they accept from the remote side of the communication.
This has practical relevance for some of the calls:
the call parameters are treated as opaque and are not
interpreted by the receiving instance. Thus, additional
information not previously foreseen and designed into the
protocol can still pass.

