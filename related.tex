%!TEX root = paper.tex
\section{Related Work}

\albert{Mention inspirations from testbeds, good software engineering principles, etc.}

\lukas{Below some abstract summaries/keywords of related works}
\lukas{Inline comments a la ``sounds interesting'' or ``not relevant'' are appreciated!}


\subsection{Overview and surveys}

\paragraph{Fog computing and its role in the Internet of things} \cite{Bonomi:2012:FCR:2342509.2342513}
Overview paper that describes fog's, key characteristics (Low latency, location awareness, Wide-spread geographical distribution, Mobility, large number of nodes, WiFi, streaming/real time applications, heterogeneity) and applications (Connected Vehicle, Smart Grid, Smart Cities, and, in general, Wireless Sensors and Actuators Networks)

\paragraph{A Survey of Fog Computing: Concepts, Applications and Issues} \cite{Yi:2015:SFC:2757384.2757397}
Overview paper about how fog and cloud are different, definitions, applications and challenges

\paragraph{A survey on IoT communication and computation frameworks: An industrial perspective}\cite{7868354}
Surveys state of the art fog computing and embedded systems platforms, reviews a high-level conceptual layered architecture for IoT from a computational perspective

\paragraph{Integration of Cloud computing and Internet of Things: A survey}\cite{botta_integration_2016}
Literature study of the CloudIoT paradigm and its applications and challenges

\paragraph{Key ingredients in an IoT recipe: Fog Computing, Cloud computing, and more Fog Computing}\cite{yannuzzi_key_2014}
Reviews technologies for IoT and Fog/Cloud applications, focuses on mobility; reliable control and actuation; and scalability

\paragraph{Fog Computing: A Taxonomy, Survey and Future Directions}\cite{mahmud_fog_2016}
Presents taxonomy of Fog computing according to the identified challenges and its key features


\subsection{Systems/Applications}

\paragraph{Fog Computing: Principles, Architectures, and Applications}\cite{dastjerdi_fog_2016}
Presents reference architecture for Fog computing and recent related development and applications

\paragraph{A gateway based fog computing architecture for wireless sensors and actuator networks \cite{7423332}}
Presents key architecture requirements: master nodes, slave nodes; virtual gateway function -, flow- and resource-management


\paragraph{Fog Computing: Platform and Applications} \cite{yi_fog_2015}
Analyzes the goals and challenges in fog computing platform, and presents platform design and prototype fog computing platform, with an example application (simple multi-party chatting application)

\paragraph{Google IoT cloud}
``https://cloudplatform.googleblog.com/2017/05/introducing-Google-Cloud-IoT-Core-for-securely-connecting-and-managing-IoT-devices-at-scale.html/'' Newly launched Google Cloud service to publish data gathered from IoT devices, e.g. sensors and process that data leveraging in-house analytic services like Google Cloud Dataflow, Google BigQuery, Google Cloud Machine Learning Engine.

\paragraph{Feasibility of Fog Computing Deployment based on Docker Containerization over RaspberryPi} \cite{bellavista_feasibility_2017}
Presents practical (as opposed to prevalent architecture and methodological approaches in the literature) experience with Kura gateway (open source IoT framework) and Docker-based containerization on RaspberryPIs

\paragraph{Evaluation of Docker as Edge computing platform}\cite{ismail_evaluation_2015}
Evaluates docker for fog computing focusing on deployment and termination, resource and service management, fault tolerance and caching

\paragraph{E-HAMC: Leveraging Fog computing for emergency alert service} \cite{7134091}
Presents service architecture for emergency alert system using smart phones. (Fog because of latency sensitivity, and to offload resource constrained devices, i.e. smart phones). The service alerts appropriate emergency department through already stored contact numbers and synchronizes fog with cloud for further analysis and safety improvement.

\paragraph{Design and Deployment of an IoT Application-Oriented Testbed}\cite{belli_design_2015}
Web of things testbed to design and evaluate services/applications in a real IoT environment solely based on standard protocols and network interfaces

\paragraph{HomeCloud: An edge cloud framework and testbed for new application delivery} \cite{pan_homecloud:_2016}
Proposes a fog framework that integrates Network Function Virtualization (NFV) and Software-Defined Networking (SDN)

\paragraph{An Open Smart City IoT Test Bed: Street Light Poles as Smart City Spines: Poster Abstract}\cite{amrutur_open_2017}
Proposes IoT testbed with modular hardware and software architecture to enable experimentation with various technologies

\paragraph{Foggy clouds and cloudy fogs: a real need for coordinated management of fog-to-cloud computing systems}\cite{masip-bruin_foggy_2016}
Presents a layered fog-to-cloud architecture to coordinate management between the fog and the cloud, e.g. for smart cities, smart transportation, and smart homes

\paragraph{Dynamic Urban Surveillance Video Stream Processing Using Fog Computing} \cite{chen_dynamic_2016}
Presents real-time information processing and decision making for urban video surveillance

\paragraph{VANET-cloud: a generic cloud computing model for vehicular Ad Hoc networks} \cite{bitam_vanet-cloud:_2015}
Reviews vehicular ad hoc network applications (traffic safety and provide computational services to road users) regarding
security and privacy, data aggregation, energy efficiency, interoperability, and resource management

\paragraph{Software defined networking-based vehicular Adhoc Network with Fog Computing} \cite{truong_software_2015}
Presents new VANET architecture that combines fog computing and software defined networks (FSDN)
Identifies components important for the system. Tackles challenges like vehicle-to-vehicle and vehicle-to-base station communication and SDN centralized control

\paragraph{A wearable augmented reality testbed for navigation and control, built solely with commercial-off-the-shelf (COTS) hardware} \cite{behringer_wearable_2000}
Presents wearable IoT testbed, designed to be worn like a jacket and used stand-alone or linked to a larger scale testbed (WIMMIS)

\paragraph{Components of fog computing in an industrial internet of things context}\cite{gazis_components_2015}
Presents Adaptive Operations Platform (AOP) for end-to-end manageability of industrial fog computing (complex physical machinery + sensors)

\paragraph{Developing IoT applications in the Fog: A Distributed Dataflow approach} \cite{giang_developing_2015}
Presents a Distributed Dataflow computing model for IoT in Fog and Cloud that eases the development of IoT/Fog applications

\paragraph{Policy-driven security management for fog computing: Preliminary framework and a case study} \cite{dsouza_policy-driven_2014}
Proposes policy-based management of resources in fog computing for secure collaboration and interoperabil{}ity between different user-requested resources in fog computing

