%!TEX root = paper.tex
\section{Introduction And Related Work}

In recent years, \gls{fc} has established itself as an approach to
Cloud Computing at the edge of the network. Fog systems are typically
characterized by the large number of geographically distributed
nodes, ranging
from embedded systems like network connected
sensors to smartphones and end-user laptops
\cite{Bonomi:2012:FCR:2342509.2342513,Yi:2015:SFC:2757384.2757397,dastjerdi_fog_2016}.
While leveraging the
ubiquitous availability and specific capabilities of such devices brings
many opportunities, it also bears new challenges unknown to traditional cloud
services. Some of these challenges include the heterogeneity of fog
nodes~\cite{Bonomi:2012:FCR:2342509.2342513,7868354,Yi:2015:SFC:2757384.2757397,mahmud_fog_2016},
their accessibility, e.g. behind private networks, and extended security and
privacy requirements \cite{botta_integration_2016}.

Also, existing proposed \gls{fc} architectures have been described
as ``siloed''~\cite{belli_design_2015}, referring to the lack of
open interfaces that characterize today's Fog devices and infrastructures.
Furthermore, opportunities and challenges have been widely addressed in the
respective literature from a theoretical point of view, but actual \gls{fc}
implementations --- let alone actively deployed and
publicly available platforms that can be used in Fog contexts --- are rare.
We review aspects of the rich existing literature below, and then
proceed to introduce our platform Seattle.


\subsection{Related Work}

Emerging technologies are a prospective area for research in \gls{fc}.
Bellavista et al.~\cite{bellavista_feasibility_2017} investigate the
applicability of Docker containers for \gls{fc}. The work uses and extends
the open-source Kura framework to create \gls{IoT} gateways that control the information
flow between Fog nodes and the Cloud, while also reviewing and benchmarking
different container related technologies used in the Fog nodes.
The focus of the work tends towards \gls{IoT} computing, where tailored services run
on Fog nodes to gather data that is forwarded to the Cloud for further processing.
A similar evaluation of Docker containers for edge computing can be found in
\cite{ismail_evaluation_2015}.

Others have contributed Fog platforms for very specific use cases, e.g.
an emergency alert system using smartphones that propagate alerts to nearby
emergency departments \cite{7134091}; an idea also treated by Masip
et al.~\cite{masip-bruin_foggy_2016}. Gazis et al. propose an ``Adaptive
Operations Platform''
to effectively apply equipment failure models in the context of Industrial
\gls{IoT} \cite{gazis_components_2015}.
Amrutur et al.\cite{amrutur_open_2017} discuss ``smart city''
applications. Vehicular use cases, including ones that are based on
ad-hoc networks, are presented by Bitam et al.~\cite{bitam_vanet-cloud:_2015}
and Truong et al.\cite{truong_software_2015}.

Apart from individual use cases, some implementations address specific individual
\gls{fc} issues, like Dsouza et al. who propose a framework for
policy management to authenticate the various actors in applications
such as smart
transportation systems \cite{dsouza_policy-driven_2014}.
Yi et al.~\cite{yi_fog_2015}, besides pointing out vendor lock-in
as a possible problem for \gls{fc}, attempt to exploit locality
in a real-world experiment with face recognition software.

In addition to academic work, commercial business solutions exist:
Cisco IOx~\cite{cisco_iox} promotes a
system that allows traditional Linux application development on
Cisco IOS powered networking devices.
Google recently released ``Google \gls{IoT} cloud'' which sells a system for
cloud based device management and a protocol bridge to connect edge nodes to
Cloud analytic systems~\cite{google_iot_core} and other infrastructure.

Finally, there are different stances on operational aspects of
\gls{fc} platforms. Some authors assume a centrally-managed and
operated approach, perhaps including monetarization~\cite{mahmud_fog_2016,7868354}, whereas others~\cite{belli_design_2015}
call for open interfaces and interoperability so that many
Fogs can coexist and provide federated services.

As will be discussed shortly, our platform Seattle offers a practical,
useable sandbox implementation that tackles the widely-acknowledged
node heterogeneity issue in \gls{fc}. Furthermore, Seattle's components
are designed for loose coupling and precise trust boundaries, so as to
enable multi-stakeholder operations. This enables deployments with
minimal mutual trust requirements, out of the box.
We have an active, live, and publicly accessible deployment of all
Seattle components described in this paper. All software is
\acrlong{FOSS}, and available from public
repositories\footnote{\url{https://github.com/SeattleTestbed}}.


\subsection{Contributions}

The contributions of this paper are as follows:
\begin{itemize}
\item We present Seattle, a practical platform for \gls{fc} research
with a real-world deployment on heterogeneous nodes,
including desktop and laptop machines, Android devices,
Raspberry Pis,
and routers and embedded devices running OpenWrt.
%Seattle nodes host Python-based \glspl{VM} that can run
%general-purpose code on a variety of platforms,
%and which self-isolate so as to not
%affect the safety and performance of the host node.
\item We further discuss how
Seattle's system architecture caters to a variety of use cases,
ranging from peer-to-peer deployments to full-fledged
provisioning by a dedicated operator, and cooperative setups,
where different stakeholders federate multiple parallel running
instances of services.
\item We present our live deployment of Seattle which has been
installed on tens of thousands of devices, and used by over 4,000
researchers and students over its 8 years of operational history.
\end{itemize}
Our deployment of Seattle has been used in multiple contexts
including research~\cite{li2015fence,rafetseder2013sensorium,zhuang2014sensibility,Eisl1010:Service,Tuts1010:Sustained,collares2011smart,zhuang2015privacy,cappos2014blursense,7133607} and education~\cite{Wallace_CCSC_2011,Cappos_CCSCCP_2010,Cappos_CCSCNW_2009,Cappos_SIGCSE_2014,Hooshangi_SIGCSE_2015};
other groups have successfully reused and augmented Seattle components
for their purposes~\cite{chard2010social,chard12ssc,caton2014social,muller2014tomato,tomato,eittenberger2012doubtless,zhuang2012distributed,zhuang2014taking,tredger2013building}.
