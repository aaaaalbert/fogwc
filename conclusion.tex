%!TEX root = paper.tex
\section{Conclusion And Outlook}

This paper presents Seattle, a practical and publicly accessible
platform for \gls{fc}. Seattle is deployed and operational in the
real world on heterogeneous nodes,
including desktop and laptop machines, Android devices,
Raspberry Pis,
and routers and embedded devices running OpenWrt.
Seattle's installer packages and sandbox implementation
thus address a widely-recognized and central issue for the
success of \gls{fc}, which is node heterogeneity.

The system architecture of Seattle consists of loosely-coupled
components. This makes the components useful as stand-alone entities,
and stimulates reuse and adaptation of existing component
implementations for unforeseen purposes.
Furthermore, since components are separated along trust boundaries,
%different implementations of many components (such as the sandbox
%or the Clearinghouse) exist. All implementations can coexist and
Seattle does not require a single centralized operator for all
components. Instead, a large range of operational scenarios with
varying scopes can be implemented, from fully local setups to
peer-to-peer resource swapping and federated multi-operator
deployments with mutually-trusted intermediaries.
Seattle's support of heterogeneous operations includes and exceeds
the capabilities of often-assumed centrally managed and
operated \gls{fc} setups.

Currently, we can report an active, live, and publicly accessible
deployment of all the Seattle components described in this paper.
More than 4,000 experimenters have used our existing deployment
to run distributed experiments, and more than 100 developers from
32 institutions
have contributed to Seattle's free, open-source software stack since
its inception in 2009.
Seattle has been installed on 40,000 devices all over the world, and
has been used
for teaching and research. Furthermore, outside groups have used existing
Seattle
components to construct new components with compatible interfaces, and
to adapt the platform to their needs.
All of Seattle's software is \acrlong{FOSS} and available at
\texttt{\url{https://github.com/SeattleTestbed}}.
